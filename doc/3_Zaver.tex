\chapter*{Závěr}
\addcontentsline{toc}{part}{Závěr}

Mým cílem bylo vytvořit hru pro VR, která by byla jednoduchá a zároveň zábavná. Ze zpětné vazby od přátel a rodiny vyplývá, že jsem v tomto ohledu uspěl. Všichni, kterým jsem hru zapůjčil, po pár minutách pochopili ovládání rozhraní a byli schopni dále hrát bez mé pomoci. Hra slouží jako skvělý vstup do světa virtuání reality těm, kteří ji nikdy neměli možnost vyzkoušet.

Od testujících jsem obdržel mnoho nápadů, jak hru dále vylepšit. Často se opakovalo téma \uv{speciální překážky}, jako propasti, pohybující se zdi aj. Tato vylepšení budou vyžadovat zásahy do algoritmu pro generování bludiště. Dalším návrhem bylo přidání režimu pro více hráčů, tedy možnost závodit s kamarádem při řešení stejného hlavolamu.

Protože se jedná o hru pro VR, vyžaduje speciální hardware pro její spuštění a tím pádem není tak přístupná, jak bych si přál. Řešením by mohlo být vytvoření verze pro Google Cardboard, čímž by hru bylo možné hrát na mobilním telefonu ve vlastnoručně vyrobeném headsetu. Jinou možností by mohlo být vytvoření AR verze (taktéž pro mobilní telefony), která nepoužívá stereoskopii.

V budoucnosti bych hru rád nabídl volně ke stažení na internetu. Zvažuji portály \textit{itch.io} a \textit{SideQuest}, které umožňují zdarma sdílet VR aplikace a hry. Další možností by mohla být \textit{Quest App Lab}, platforma provozovaná přímo společností Meta, ze které lze získat nezávisle vyvíjený software pro Meta Quest bez nutnosti vlastnit vývojářský účet.