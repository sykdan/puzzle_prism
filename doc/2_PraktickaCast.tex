\part{Praktická část}

\chapter{Cíl projektu}

V praktické části mé práce jsem vytvořil vlastní hru pro VR. Stanovil jsem si následující kritéria:

\begin{itemize}
	\item \textbf{Nenáročnost}. Chtěl jsem, aby má hra nebyla obtížná na pochopení. Většina VR her vyžaduje zvýšenou fyzickou aktivitu nebo hlubší porozumění VR konceptů. Ideálně jsem chtěl vytvořit hru, kterou bych mohl komukoliv půjčit, aniž bych musel dlouze vysvětlovat její princip.
  \item \textbf{Grafická jednoduchost}. Nepovažuji se za umělce a umím vytvářet pouze základní modely a textury. Bylo pro mě tedy klíčové přijít s takovým konceptem, který by byl vzhledově nenáročný.
  \item \textbf{Interaktivita}. Od své hry jsem chtěl, aby doopravdy využívala funkce, které VR platformy poskytují. Tedy ovládání pohybem, jednoduchá fyzika atp.
\end{itemize}

Dospěl jsem k následujícímu návrhu: Hra sestává z kostky tvořené úrovněmi nakládanými na sebe. Každá úroveň představuje bludiště, skrze které musí nakláněním hráč navigovat kuličku. Při dosažení cíle je úroveň odebrána, kostka se zmenší a je odhalena další úroveň.

To splňuje moje kritéria - hra je jednoduchá, na vykreslení potřebuje jen geometrické tvary a vyžaduje pohyb rukama pro naklánění bludiště.

Již od samého začátku jsem svou hru chtěl napsat v herním enginu. Herních enginů zdarma, které zároveň podporují XR, není mnoho. Zvolil jsem si svobodný Godot Engine.

\chapter{Herní engine Godot}

Godot je bezplatný herní engine. Hlavní důvod, proč jsem si ho vybral namísto známějšího Unity, je jeho licence. Godot je šířen pod svobodnou licencí MIT, která vývojářům dovoluje kód používat komerčně i naopak, a to pod jedinou podmínkou: text licence je ve výsledné práci zachován. Unity Engine je naopak distribuován pod nesvobodnou licencí a za některé funkce musí vývojáři platit. Godot se v poslední době stává více a více populárním a obdržel investice od velkých společností. Mezi ně patří např. grant od Epic Games pro vývoj grafiky a grant od Mety pro vývoj XR funkcí. \cite{godot_epicgames} \cite{godot_meta}

\section{Struktura Godot projektu}

Pro úpravu Godot projektů používáme oficiální editor. Základními stavebními bloky enginu jsou uzly, které jsou uspořádány do stromu, podobně jako jsou webové stránky reprezentovány stromem značek. Tyto uzly samy o sobě nemají mnoho funkcí, jejich kombinací ale můžeme dosáhnout komplexnějšího chování. Jednotlivé uzly se dají přirovnat k třídám v objektově orientovaném programování, které se navzájem rozšiřují.

Například uzel Node3D disponuje atributem \texttt{position} (pozice), ale uzel Camera3D, který rozšiřuje Node3D, disponuje atributy \texttt{position} a \texttt{fov} (field of view; zorný úhel). Dceřinné uzly zároveň dědí atributy rodičovských uzlů, pokud oba rozšiřují společný uzel (pokud je např. Camera3D dceřinný uzel Node3D, pak se Camera3D pohybuje společně s Node3D (sdílí atribut \texttt{position}) a \texttt{position} na Camera3D je relativní posun od pozice Node3D)

Uzlům můžeme přiřadit \textit{skript}, který upravuje jejich chování. Tento skript se chová jako třída v OOP, která rozšiřuje daný uzel. Pokud rozšiřujeme uzel pro 2D grafiku, můžeme metodou získat pozici jako 2D vektor. Pokud rozšiřujeme uzel pro 3D grafiku, stejnou metodou můžeme získat pozici jako 3D vektor. Stejně tak můžeme přepisovat virtuální funkce. Projekty v Godotu používají dedikovaný jazyk zvaný GDScript, existuje ale oficiální podpora jazyka C\# a komunitně vytvořená rozšíření pro Rust.

Uzel a jeho dceřinné uzly společně s přiřazenými skripty můžeme dále uložit jako izolovanou \textit{scénu} (textový soubor .tscn), kterou můžeme poté instancovat. Scéna se pak chová jako samostatný uzel. Jedna scéna je vždy hlavní scénou a je otevřena při spuštění projektu.

\begin{figure}[H]
  \centering
  \includegraphics[height=180pt]{godot_editor.png}
  \caption{Editor Godot Engine a otevřená scéna Maze.tscn}
  \label{godot_editor_maze_tscn}
\end{figure}

!!! Dopsat strukturu !!!

\section{Jazyk \textit{GDScript}}

GDScript je jazyk, ve kterém se vyvíjí většina Godot projektů. Můj projekt je taktéž psán v GDScriptu a z tohoto důvodu jej zde popisuji pro snadnější pochopení kódu.

GDScript je vysokoúrovňový, objektově orientovaný, imperativní a hybridně (staticky i dynamicky) typovaný jazyk. Vzhledově nápadně připomíná jazyk Python a podobně jako Python používá indentaci pro vyjádření řídící struktury. Nabízí širokou škálu vestavěných tříd pro manipulaci s hodnotami typickými pro hry, matematiku a fyziku (textury, vektory, matice) a obecné typy (celé číslo, desetinné číslo, textový řetězec atd.). Proměnné programátor může anotovat typem. Podporuje koprogramy (coroutines) a událostmi řízené programování pomocí "signálů". Pro svou hlubokou integraci se samotným enginem je vhodný pro jednoduché projekty, kvůli své dynamičnosti ovšem nenabízí stejnou rychlost jako kompilované jazyky. \cite{gdscript_reference}

Níže je příklad kódu v GDScriptu.

\lstinputlisting{code/sample_gdscript.gd}

\chapter{Algoritmy}

\section{Generování bludiště}

\chapter{Interaktivita}

\section{Manipulace s bludištěm}

\section{Uživatelské rozhraní}
