\chapter*{Úvod}
\addcontentsline{toc}{chapter}{Úvod}

Pojmem \gls{xr} (eXtended Reality; rozšířená realita) rozumíme libovolnou technologii, která nějakým způsobem upravuje realitu vnímanou člověkem. Její uplatnění je různé -- od her a virtuálních schůzek až po simulace lékařských zákroků a raketových letů. \cite{muni_kybernetika}

Nějakou formu \gls{xr} můžeme využívat na skoro každém moderním zařízení. Na mobilních zařízení augmentovanou realitu, na počítači s připojeným headsetem virtuální realitu a na speciálních samostatných (stand-alone) headsetech virtuální i smíšenou realitu. 

Tato práce popisuje historický vývoj \gls{xr} a její stav v přítomnosti. Dále popisuje technologie, které \gls{xr} aplikace využívají, po stránce hardwaru i softwaru. V praktické části se zaměřuji na \gls{vr}, s cílem vytvořit vlastní hru pro virtuální realitu za pomoci současných nástrojů.
