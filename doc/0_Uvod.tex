\chapter*{Úvod}
\addcontentsline{toc}{chapter}{Úvod}
Rozšířená realita se v~dnešní době stává čím dál častěji zmiňovaným tématem. \gls{xr} zařízení se stávají čím dál více dostupnějšími a v~médiích často slýcháme o~pojmu \textit{Metaverse} -- platformě, o~kterou se technologické společnosti přou.

Pojmem \gls{xr} (eXtended Reality; rozšířená realita) rozumíme libovolnou technologii, která nějakým způsobem upravuje realitu vnímanou člověkem. Její uplatnění je různé -- od videoher a virtuálních schůzek až po simulace lékařských zákroků a raketových letů. \cite{muni_kybernetika}

Nějakou formu \gls{xr} můžeme využívat na skoro každém moderním zařízení. Na mobilních zařízení augmentovanou realitu, na počítači s~připojeným headsetem virtuální realitu a na speciálních samostatných (stand-alone) headsetech virtuální i smíšenou realitu.

V~práci popisuji historický vývoj \gls{xr} a její stav v~přítomnosti. Dále popisuje technologie a specifický hardware, na kterých je \gls{xr} založena, a přibližuje softwarové platformy a rozhraní, které současné XR aplikace používají. V~praktické části se zaměřuji na \gls{vr} a využití současných vývojových nástrojů, s~cílem vytvořit jednoduchou a uživatelsky přístupnou hru pro virtuální realitu.