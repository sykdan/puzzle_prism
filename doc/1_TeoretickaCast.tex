\part{Teoretická část}

\chapter{Historie \gls{xr}}

\section{Počátky \gls{xr}}

První pokusy o rozšířenou realitu pochází už z 50. let 20. století, kdy Morton Hellig, americký kinematograf, přivádí na svět své zařízení zvané Sensorama. Nejednalo se ovšem o headset, které si vybavíme dnes -- Sensorama bylo statické zařízení, vzhledově připomínající spíše arkádový automat. Hellig toto zařízení nazýval "zážitkovým divadlem", které bylo schopno zobrazit 3D obraz, pouštět stereo zvuk a vytvářet vítr. Tím se od dnešní XR technologie zásadně liší; nepřijímá vstup uživatele. Sensorama se ovšem nedočkala úspěchu a známe ji jen jako historicky první pokusy o virtuální realitu.

\begin{figure}[H]
    \centering
    \includegraphics[height=250pt]{sensorama.png}
    \caption{Sensorama}
    \label{sensorama}
\end{figure}

Dalším průkopníkem rozšířené reality je Ivan Sutherland, americký vědec, který je často označován jako otec počítačové grafiky. Ve svém díle The Ultimate display (ultimátní displej) popsal virtuální realitu tak, jak ji známe dnes. Virtuální realitu si představoval jako helmu, do které odesílá obraz počítač v reálném čase. Uživatel se tak měl ocitnout ve fiktivním světě nerozpoznatelným od světa reálného.

Tuto představu se Sutherland snažil realizovat společně se svými studenty. Vynalezli vůbec první headset pro virtuální realitu, zvané The Sword of Damocles -- tedy Damoklův meč. Vzhledem k jeho primitivnosti zobrazovalo pouze čtvercové místnosti tvořené z čar, které software následně transofrmoval do správné perspektivy. Pohyby sledovalo pomocí mechanického ramene připevněného ke stropu, ze kterého byl headset zavěšený.

\section{\gls{xr} dnes}

lorem ipsum

\chapter{Hardware}
lorem ipsum

něco o trackování, výrobcích, atd. zmínit trackovací systém bez věží od Meta/FB

rozlišit 6dof vs 3dof

\chapter{Software}
lorem ipsum

zmínit OpenXR, WebXR jakožto API co pohánějí XR

rozlišit PCVR na windows a mobilní VR na bázi androidu