\documentclass[12pt]{report}
\usepackage{Styl}			

% Úvvodní strana
\author{Daniel Sýkora}
\title{Rozšířená realita}
\date{14. února 2024}
\vedouci{Dr.rer.nat. Michal Kočer}
\place{v Českých Budějovicích}
\skolnirok{2023/2024}
\logo{\includegraphics{GJ8_logotyp.pdf}}

\begin{document}

\mytitlepage
\prohlaseni
{
	Prohlašuji, že jsem tuto práci vypracoval samostatně s vyznačením všech použitých pramenů.
}
\abstrakt
{ % Abstrakt
	\lipsum[1]
}
{ % Klíčová slova
	\begin{itemize}
		\item Headset -- Označuje nejčastěji zařízení, které si uživatel nasadí na hlavu pro vstup do \gls{xr}.
				Nejčastěji se skládá z dvou displejů pro stereoskopické zobrazení, senzorů pro sledování pohybu,
				reproduktorů, mikrofonu atd.

		\item Virtuální realita (\gls{vr}) -- Označuje typ rozšířené reality, ve kterém je uživatel pomocí headsetu
				kompletně přenesen do virtuálního světa, prvky z fyzického světa vůbec nevidí a neinteraguje s nimi.

		\item Upravená (či augmentovaná) realita (\gls{ar}) -- Označuje typ rozšířené reality, ve kterém uživatel interaguje
				s virtuálními prvky v kontextu fyzického světa (např. zobrazení virtuálního předmětu na stole v životní velikosti).
				Uživatel však nemá nasazený headset a virtuální prvky vidí jen zčásti (např. na obrazovce mobilního telefonu).

		\item Smíšená realita (\gls{mr}) -- Označuje typ rozšířené reality, ve kterém uživatel interaguje s virtuálními předměty
				v kontextu fyzického světa, přičemž má nasazený headset a virtuální prvky vyplňují celé zorné pole.

		\item Rozšířená realita (\gls{xr}) -- Tento pojem zastřešuje \gls{vr}, \gls{ar} a \gls{mr}.
	\end{itemize}
}
\podekovani
{
	Děkuji panu Michalu Kočerovi za vedení mé práce a všem mým přátelům, kteří otestovali moji hru a poskytli mi zpětnou vazbu.
}

\tableofcontents
\newpage

\chapter*{Úvod}
\addcontentsline{toc}{chapter}{Úvod}

Pojmem \gls{xr} (eXtended Reality; rozšířená realita) rozumíme libovolnou technologii, která nějakým způsobem upravuje realitu vnímanou člověkem. Její uplatnění je různé -- od her a virtuálních schůzek až po simulace lékařských zákroků a raketových letů. \cite{muni_kybernetika}

Nějakou formu \gls{xr} můžeme využívat na skoro každém moderním zařízení. Na mobilních zařízení augmentovanou realitu, na počítači s připojeným headsetem virtuální realitu a na speciálních samostatných (stand-alone) headsetech virtuální i smíšenou realitu. 

Tato práce popisuje historický vývoj \gls{xr} a její stav v přítomnosti. Dále popisuje technologie, které \gls{xr} aplikace využívají, po stránce hardwaru i softwaru. V praktické části se zaměřuji na \gls{vr}, s cílem vytvořit vlastní hru pro virtuální realitu za pomoci současných nástrojů.

\part{Teoretická část}

\chapter{Historie \gls{xr}}

\section{Počátky \gls{xr}}

První pokusy o rozšířenou realitu pochází už z 50. let 20. století, kdy Morton Hellig, americký kinematograf, přivádí na svět své zařízení zvané Sensorama. Nejednalo se ovšem o headset, které si vybavíme dnes -- Sensorama bylo statické zařízení, vzhledově připomínající spíše arkádový automat. Hellig toto zařízení nazýval "zážitkovým divadlem", které bylo schopno zobrazit 3D obraz, pouštět stereo zvuk a vytvářet vítr. Tím se od dnešní XR technologie zásadně liší; nepřijímá vstup uživatele. Sensorama se ovšem nedočkala úspěchu a známe ji jen jako historicky první pokusy o virtuální realitu.

\begin{figure}[H]
    \centering
    \includegraphics[height=250pt]{sensorama.png}
    \caption{Sensorama}
    \label{sensorama}
\end{figure}

Ivan Sutherland, americký vědec, který je často označován jako průkopník v oblasti počítačové grafiky. Společně se svými studenty vynalezl vůbec první zařízení pro virtuální realitu -- zvané Damoklův meč. Vzhledem jeho primitivnosti zobrazovalo pouze čtvercové místnosti tvořené z čar, které software následně transofrmoval do správné perspektivy. Pohyby sledovalo pomocí mechanické paže připevněné ke stropu, ke které byl headset připevněný.

\section{\gls{xr} dnes}

lorem ipsum

\chapter{Hardware}
lorem ipsum

něco o trackování, výrobcích, atd. zmínit trackovací systém bez věží od Meta/FB

rozlišit 6dof vs 3dof

\chapter{Software}
lorem ipsum

zmínit OpenXR, WebXR jakožto API co pohánějí XR

rozlišit PCVR na windows a mobilní VR na bázi androidu
\part{Praktická část}

\chapter{Cíl projektu}

\chapter{Herní engine Godot}

\section{Struktura Godot projektů}

\section{Jazyk \textit{GDScript}}

\chapter{Algoritmy}

\section{Generování bludiště}

\chapter{Interaktivita}

\section{Manipulace s bludištěm}

\section{Uživatelské rozhraní}

\chapter*{Závěr}
\addcontentsline{toc}{part}{Závěr}

\lipsum[1]

\nocite{*}

\appendix
\addcontentsline{toc}{part}{Apendix}

\printbibliography
\addcontentsline{toc}{chapter}{Bibliografie}

\setglossarystyle{list}
\printglossary[title={Zkratky}]

\listoffigures
\addcontentsline{toc}{chapter}{\listfigurename}

\end{document}
