\documentclass[12pt]{report}
\usepackage{Styl}			

% Úvodní strana
\author{Daniel Sýkora}
\title{Rozšířená realita}
\date{14. února 2024}
\vedouci{Dr.rer.nat. Michal Kočer}
\place{v~Českých Budějovicích}
\skolnirok{2023/2024}

\logo{\includegraphics{GJ8_logotyp.pdf}}

\begin{document}

\mytitlepage
\prohlaseni
{
	Prohlašuji, že jsem tuto práci vypracoval samostatně s~vyznačením všech použitých pramenů.
}
\abstrakt
{ % Abstrakt
	Tato práce se zabývá širokým tématem \gls{xr} a~jejího využití. V~první části přibližuje historii \gls{xr} a~její zpracování po technické stránce, na úrovni hardwaru i~softwaru. V~druhé části se zaměřuje na tvorbu videohry pro moderní \gls{vr} zařízení a~popisuje výzvy spojené s~vývojem videoher a~zvláštnostmi platformy \gls{xr}, které musí být při vývoji zohledněny.
}
{ % Klíčová slova
	\textit{Headset (HMD), Rozšířená realita, Virtuální realita, Augmentovaná realita, Smíšená realita, sledování pohybu (motion tracking), herní engine}
}
\podekovani
{
	Děkuji panu Michalovi Kočerovi za vedení mé práce a~zpětnou vazbu. Také bych rád poděkoval rodině, přátelům a~spolužákům, kteří testovali moji hru a~poskytli mi na k~ní své nápady a~připomínky.
}

\tableofcontents
\newpage

\chapter*{Úvod}
\addcontentsline{toc}{chapter}{Úvod}

Pojmem \gls{xr} (eXtended Reality; rozšířená realita) rozumíme libovolnou technologii, která nějakým způsobem upravuje realitu vnímanou člověkem. Její uplatnění je různé -- od her a virtuálních schůzek až po simulace lékařských zákroků a raketových letů. \cite{muni_kybernetika}

Nějakou formu \gls{xr} můžeme využívat na skoro každém moderním zařízení. Na mobilních zařízení augmentovanou realitu, na počítači s připojeným headsetem virtuální realitu a na speciálních samostatných (stand-alone) headsetech virtuální i smíšenou realitu. 

Tato práce popisuje historický vývoj \gls{xr} a její stav v přítomnosti. Dále popisuje technologie, které \gls{xr} aplikace využívají, po stránce hardwaru i softwaru. V praktické části se zaměřuji na \gls{vr}, s cílem vytvořit vlastní hru pro virtuální realitu za pomoci současných nástrojů.

\part{Teoretická část}

\chapter{Historie \gls{xr}}

\section{Počátky \gls{xr}}

První pokusy o rozšířenou realitu pochází už z 50. let 20. století, kdy Morton Hellig, americký kinematograf, přivádí na svět své zařízení zvané Sensorama. Nejednalo se ovšem o headset, které si vybavíme dnes -- Sensorama bylo statické zařízení, vzhledově připomínající spíše arkádový automat. Hellig toto zařízení nazýval "zážitkovým divadlem", které bylo schopno zobrazit 3D obraz, pouštět stereo zvuk a vytvářet vítr. Tím se od dnešní XR technologie zásadně liší; nepřijímá vstup uživatele. Sensorama se ovšem nedočkala úspěchu a známe ji jen jako historicky první pokusy o virtuální realitu.

\begin{figure}[H]
    \centering
    \includegraphics[height=250pt]{sensorama.png}
    \caption{Sensorama}
    \label{sensorama}
\end{figure}

Ivan Sutherland, americký vědec, který je často označován jako průkopník v oblasti počítačové grafiky. Společně se svými studenty vynalezl vůbec první zařízení pro virtuální realitu -- zvané Damoklův meč. Vzhledem jeho primitivnosti zobrazovalo pouze čtvercové místnosti tvořené z čar, které software následně transofrmoval do správné perspektivy. Pohyby sledovalo pomocí mechanické paže připevněné ke stropu, ke které byl headset připevněný.

\section{\gls{xr} dnes}

lorem ipsum

\chapter{Hardware}
lorem ipsum

něco o trackování, výrobcích, atd. zmínit trackovací systém bez věží od Meta/FB

rozlišit 6dof vs 3dof

\chapter{Software}
lorem ipsum

zmínit OpenXR, WebXR jakožto API co pohánějí XR

rozlišit PCVR na windows a mobilní VR na bázi androidu
\part{Praktická část}

\chapter{Cíl projektu}

\chapter{Herní engine Godot}

\section{Struktura Godot projektů}

\section{Jazyk \textit{GDScript}}

\chapter{Algoritmy}

\section{Generování bludiště}

\chapter{Interaktivita}

\section{Manipulace s bludištěm}

\section{Uživatelské rozhraní}

\chapter*{Závěr}
\addcontentsline{toc}{part}{Závěr}

Můj projekt dopadl velmi dobře a splňuje předpoklady, které jsem si stanovil. Výsledná hra je jednoduchá a příjemená na hraní a slouží jako skvělý vstup do světa virtuání reality těm, kteří ji neměli možnost vyzkoušet.

V budoucnu bych hru rád nabídl volně ke stažení na internetu. Nabízí se portály \textit{itch.io} nebo \textit{SideQuest}, které umožňují zdarma sdílet VR aplikace a hry. Další možností je \textit{Quest App Lab}, platforma provozovaná přímo společností Meta, ze které lze instalovat nezávisle vyvíjený software bez nutnosti vlastnit vývojářský účet.

!! DOPSAT !!

\nocite{*}

\addcontentsline{toc}{part}{Přílohy}
\addcontentsline{toc}{chapter}{Bibliografie}
\printbibliography

\setglossarystyle{list}
\printglossary[title={Zkratky}]

\listoffigures
\addcontentsline{toc}{chapter}{\listfigurename}

\appendix

\chapter{Ukázky ze hry}\label{apx_screenshots}
Dostupné také jako video na YouTube na adrese \url{https://youtu.be/SBnWp2jJ9SY}.
\begin{figure}[H]
	\centering
	\includegraphics[width=0.98\linewidth]{screenshot1.png}
	\includegraphics[width=0.98\linewidth]{screenshot2.png}
\end{figure}
\begin{figure}[H]
	\centering
	\includegraphics[width=0.98\linewidth]{screenshot3.png}
	\includegraphics[width=0.98\linewidth]{youtubethumbnail.png}
\end{figure}

\chapter{Základní ukázka GDScript kódu}\label{apx_gscript_sample}
\lstinputlisting{code/sample_gdscript.gd}

\chapter{Ovládání pohybem obou rukou}\label{apx_gripped_object_transformation}
Tento kód z~velké části využívá faktu, že uzly v~Godotu dědí své transformace. Pokud se pohne rodičovský uzel, dceřinný uzel se pohne společně s~ním, a~rozdíl pozice a~rotace mezi nimi zůstane stejný. Toto chování je ovšem ignorováno v~instrukcích, které mění hodnoty atributů \texttt{global\_*}.

Většina pohybem ovládaných uzlů má proto jeden dceřinný uzel (v~kódu nazván jako \textit{Origin}), který je při uchopení přemístěn úpravou \texttt{global\_transform} tak, aby umístěním splýval s~uchyceným předmětem (toto v~přiloženém kódu není obsaženo). Uchycený předmět je poté synchronizován s~pozicí a~rotací tohoto \textit{Origin}u. Není tak potřeba manuálně počítat rozdíl transformace.

\lstinputlisting{code/gripped_object_transformation.gd}

\chapter{Wilsonův algoritmus v~GDScript}\label{apx_mazegen}
\lstinputlisting{code/mazegen.gd}

\end{document}
