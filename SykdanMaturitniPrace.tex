\documentclass[12pt]{report}			
\usepackage{MP}							

\author{Daniel Sýkora}
\title{Rozšířená realita (XR)}
\date{14. února 2024}
\vedouci{Dr.rer.nat. Michal Kočer}
\place{V Českých Budějovicích}
\skolnirok{2023/2020}
\logo{
	\includegraphics[scale=0.75]{logo.png}
}

\begin{document}

\mytitlepage

\prohlaseni{
	Prohlašuji, že jsem tuto práci vypracoval samostatně s vyznačením všech použitých pramenů.
}

\abstrakt{
	\lipsum[1]
}{
	\lipsum[1]
}

\podekovani{
	\lipsum[2]
}

\tableofcontents
\newpage




\chapter*{Úvod}

\lipsum[1]


\part{Teoretická část}

\chapter{Asymptotická notace}

\section{O-notace}
Odkaz v závorekách: \cite[see][page 900]{einstein}\\
Odkaz: \cite{knuthwebsite}\\
A odkaz pod čarou: \footcite[see][s. 42]{latexcompanion}\\
Dobrý den, ahoj, \gls{atd}\\
Praha, \gls{tj} hlavní město ČR
$\xi = 3.14 = \pi$

\section{Abeceda Abeceda Abeceda Abeceda Abeceda Abeceda Abeceda Abeceda Abeceda Abeceda }
\begin{figure}
	\includegraphics[width=\linewidth]{test.jpg}
	\caption{Testovací}
	\label{fig:test}
\end{figure}
\begin{table}
	\caption{Testovací}
	\label{tab:test2}
	\begin{tabular}{ccccc}
		1 & 1 & 1  & 1  & 1  \\
		1 & 2 & 3  & 4  & 5  \\
		1 & 3 & 6  & 10 & 15 \\
		1 & 4 & 10 & 30 & 45
	\end{tabular}
\end{table}

Obrázek \ref{fig:test} ukazuje Shangai z Pixabay.\\
Tabulka \ref{tab:test2} ukazuje hádejte, co.

\part{Praktická část}



\appendix
\addcontentsline{toc}{part}{Apendix}

\chapter*{Závěr}

\lipsum[1]

\nocite{*}
\printbibliography
\addcontentsline{toc}{chapter}{Bibliografie}
\printglossary[title={Zkratky}]
\listoffigures
\listoftables

\begin{prilohy}
	\pitem{Fotky z pokusů}
	\eitem{Vlastní program}
	\eitem{Dokumentace}
	\eitem{Testovací data}
\end{prilohy}
\end{document}